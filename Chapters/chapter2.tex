\chapter{Title Chapter 2}
\label{ch:chapter2} 

    Here there are a couple of formulas. 
    The $SU(2)$ doublet 

			\begin{equation}
			\label{eq:Higgs_field}
				\phi = 
				\begin{pmatrix}
					\phi^+ \\ \phi^0
				\end{pmatrix} 
			\end{equation}

			\noindent with $\phi^+$ and $\phi^0$ generic complex fields: 

			\begin{equation}
				\phi^+ = \frac{\phi_1 + i \phi_2}{\sqrt{2}},  \qquad \phi^0 = \frac{\phi_3 + i \phi_4}{\sqrt{2}}
			\end{equation}

			\noindent Consider a Lagrangian of the form: 

			\begin{equation}
			\label{eq:Higgs_lagrangian}
				\mathcal{L_{\mathrm{Higgs}}} = ( \partial_{\mu} \phi )^* \left ( \partial^{\mu} \phi \right ) - V(\phi)
			\end{equation}

			\noindent where $V(\phi)$ is now the Higgs potential. Re-normalisability and $SU(2)_L \otimes U(1)_Y$ invariance require the Higgs potential to be of the following form: 

			\begin{equation}
			\label{eq:Higgs_potential}
				V(\phi) = \mu^2  \phi^\dagger \phi + \lambda \left ( \phi^\dagger \phi \right )^2 
			\end{equation}

			\noindent The Lagrangian in Equation~\ref{eq:Higgs_lagrangian} is the Higgs Lagrangian if $\phi$ is chosen to be the following:

			\begin{equation*}
				\phi = 
				\begin{pmatrix}
					\phi^+ \\ \phi^0
				\end{pmatrix} 
				=
				\begin{pmatrix}
					G^+ \\ \frac{1}{\sqrt{2}} \left ( v + H + iG^0 \right )
				\end{pmatrix}
			\end{equation*}
